% This template is borrowed from the Reed College LaTeX thesis template. Most of the work
% for the document class was done by Sam Noble (SN), as well as this
% template. Later comments etc. by Ben Salzberg (BTS). Additional
% restructuring and APA support by Jess Youngberg (JY).
% Your comments and suggestions are more than welcome; please email
% them to cus@reed.edu
%
% See http://web.reed.edu/cis/help/latex.html for help. There are a
% great bunch of help pages there, with notes on
% getting started, bibtex, etc. Go there and read it if you're not
% already familiar with LaTeX.
%
% Any line that starts with a percent symbol is a comment.
% They won't show up in the document, and are useful for notes
% to yourself and explaining commands.
% Commenting also removes a line from the document;
% very handy for troubleshooting problems. -BTS

% As far as I know, this follows the requirements laid out in
% the 2002-2003 Senior Handbook. Ask a librarian to check the
% document before binding. -SN

%%
%% Preamble
%%
% \documentclass{<something>} must begin each LaTeX document
\documentclass[12pt,twoside]{deuthesis}
% Packages are extensions to the basic LaTeX functions. Whatever you
% want to typeset, there is probably a package out there for it.
% Chemistry (chemtex), screenplays, you name it.
% Check out CTAN to see: http://www.ctan.org/
%%
\usepackage{graphicx,latexsym}
\usepackage{amsmath}
\usepackage{amssymb,amsthm}
\usepackage{longtable,booktabs,setspace}
\usepackage{chemarr} %% Useful for one reaction arrow, useless if you're not a chem major
\usepackage[hyphens]{url}
% Added by CII
\usepackage{hyperref}
\usepackage{lmodern}
\usepackage{float}
\floatplacement{figure}{H}
% End of CII addition
\usepackage{rotating}

% Next line commented out by CII
%%% \usepackage{natbib}
% Comment out the natbib line above and uncomment the following two lines to use the new
% biblatex-chicago style, for Chicago A. Also make some changes at the end where the
% bibliography is included.
%\usepackage{biblatex-chicago}
%\bibliography{thesis}


% Added by CII (Thanks, Hadley!)
% Use ref for internal links
\renewcommand{\hyperref}[2][???]{\autoref{#1}}
\def\chapterautorefname{Chapter}
\def\sectionautorefname{Section}
\def\subsectionautorefname{Subsection}
% End of CII addition

% Added by CII
\usepackage{caption}
\captionsetup{width=5in}
% End of CII addition

% \usepackage{times} % other fonts are available like times, bookman, charter, palatino

% Syntax highlighting #22

% To pass between YAML and LaTeX the dollar signs are added by CII
\title{PROJE BAŞLIĞI}
%\author{Pelin PEKERMerve AKEdanur Binnaz DURSUNAhmet ÇALI} %Tek yazar için
\author{Pelin PEKER \\ Merve AK \\ Edanur Binnaz DURSUN \\ Ahmet ÇALI} %Çok yazar için
% The month and year that you submit your FINAL draft TO THE LIBRARY (May or December)
\date{May 2024}
\division{İSTATİSTİK BÖLÜMÜ}
\advisor{Dr.~Engin YILDIZTEPE}
\institution{FEN FAKÜLTESİ}
\degree{Bitirme Projesi Raporu}
%If you have two advisors for some reason, you can use the following
% Uncommented out by CII
% End of CII addition

%%% Remember to use the correct department!
\department{İstatistik Bölümü}
% if you're writing a thesis in an interdisciplinary major,
% uncomment the line below and change the text as appropriate.
% check the Senior Handbook if unsure.
%\thedivisionof{The Established Interdisciplinary Committee for}
% if you want the approval page to say "Approved for the Committee",
% uncomment the next line
%\approvedforthe{Committee}

% Added by CII
%%% Copied from knitr
%% maxwidth is the original width if it's less than linewidth
%% otherwise use linewidth (to make sure the graphics do not exceed the margin)
\makeatletter
\def\maxwidth{ %
  \ifdim\Gin@nat@width>\linewidth
    \linewidth
  \else
    \Gin@nat@width
  \fi
}
\makeatother

\renewcommand{\contentsname}{Table of Contents}
% End of CII addition

\setlength{\parskip}{0pt}

% Added by CII

\providecommand{\tightlist}{%
  \setlength{\itemsep}{0pt}\setlength{\parskip}{0pt}}

\Acknowledgements{
Tüm çalışma süresince yönlendiriciliği, katkıları ve yardımları ile yanımızda olan danışmanımız Dr.~Engin YILDIZTEPE 'ye ve böyle bir çalışmayı yapmamız için bize fırsat tanıyan Dokuz Eylül Üniversitesi Fen Fakültesi İstatistik Bölümüne teşekkür ederiz.\\
\strut \\
\strut \\
Pelin PEKER\\
Merve AK\\
Edanur Binnaz DURSUN\\
Ahmet ÇALI\\
}

\Dedication{

}

\Preface{
``PROJE BAŞLIĞI'' başlıklı bitirme projesi raporu tarafıgitmdan okunmuş, kapsamı ve niteliği açısından bir Bitirme Projesi raporu olarak kabul edilmiştir.\\
\strut \\
\strut \\
Dr.~Engin YILDIZTEPE
}

\AbstractTR{
Değişim noktası verilerde meydana gelen beklenmedik değişiklikler olarak tanımlanabilir. Değişim noktası tespit yöntemleri bu noktaları istatistiksel tekniklerle bulmayı amaçlar. Değişim noktası analizi finans, kalite kontrol, ağ analizi gibi çok farklı alanlarda kullanılmaktadır. Bu çalışmada değişim noktası tespit yöntemleri incelenmiştir. Çalışma kapsamında AMOC, BinSeg, Parçalı Regresyon, Pelt ve Prophet algoritmaları kullanılmıştır. Algoritmaların uygulamadaki performanslarını belirlemek amacıyla yirmi yapay ve on bir gerçek veri kullanılmıştır. Algoritmaların performansları F1 puanı ve kapsama ölçütü kullanılarak değerlendirilmiştir. Değişim noktası içeren yapay veri üretmek ve bahsedilen algoritmaları uygulayabilmek amacıyla bir RShiny web uygulaması geliştirilmiştir. Çalışmada R ve Python programlama dilleri kullanılmıştır.\\
\strut \\

\textbf{Anahtar Kelimeler:} Değişim noktası, AMOC, BinSeg, Pelt, Prophet, Parçalı Regresyon, R Shiny
}

\Abstract{
\textbf{Keywords:} keyword1, keyword2, keyword3
}


% End of CII addition
%%
%% End Preamble
%%
%

\begin{document}

% Everything below added by CII
  \maketitle

\frontmatter % this stuff will be roman-numbered
\pagestyle{empty} % this removes page numbers from the frontmatter

\begin{preface}
	``PROJE BAŞLIĞI'' başlıklı bitirme projesi raporu tarafıgitmdan okunmuş, kapsamı ve niteliği açısından bir Bitirme Projesi raporu olarak kabul edilmiştir.\\
\strut \\
\strut \\
Dr.~Engin YILDIZTEPE
\end{preface}

  \begin{acknowledgements}
    Tüm çalışma süresince yönlendiriciliği, katkıları ve yardımları ile yanımızda olan danışmanımız Dr.~Engin YILDIZTEPE 'ye ve böyle bir çalışmayı yapmamız için bize fırsat tanıyan Dokuz Eylül Üniversitesi Fen Fakültesi İstatistik Bölümüne teşekkür ederiz.\\
    \strut \\
    \strut \\
    Pelin PEKER\\
    Merve AK\\
    Edanur Binnaz DURSUN\\
    Ahmet ÇALI\\
  \end{acknowledgements}

\begin{abstractTR}
	Değişim noktası verilerde meydana gelen beklenmedik değişiklikler olarak tanımlanabilir. Değişim noktası tespit yöntemleri bu noktaları istatistiksel tekniklerle bulmayı amaçlar. Değişim noktası analizi finans, kalite kontrol, ağ analizi gibi çok farklı alanlarda kullanılmaktadır. Bu çalışmada değişim noktası tespit yöntemleri incelenmiştir. Çalışma kapsamında AMOC, BinSeg, Parçalı Regresyon, Pelt ve Prophet algoritmaları kullanılmıştır. Algoritmaların uygulamadaki performanslarını belirlemek amacıyla yirmi yapay ve on bir gerçek veri kullanılmıştır. Algoritmaların performansları F1 puanı ve kapsama ölçütü kullanılarak değerlendirilmiştir. Değişim noktası içeren yapay veri üretmek ve bahsedilen algoritmaları uygulayabilmek amacıyla bir RShiny web uygulaması geliştirilmiştir. Çalışmada R ve Python programlama dilleri kullanılmıştır.\\
\strut \\

\textbf{Anahtar Kelimeler:} Değişim noktası, AMOC, BinSeg, Pelt, Prophet, Parçalı Regresyon, R Shiny
\end{abstractTR}

\begin{abstract}
	\textbf{Keywords:} keyword1, keyword2, keyword3
\end{abstract}


  \hypersetup{linkcolor=black}
  \setcounter{tocdepth}{2}
  \tableofcontents

  \listoftables

  \listoffigures


% This was added by EY

\mainmatter % here the regular arabic numbering starts
\pagestyle{fancyplain} % turns page numbering back on

\hypertarget{thesisdownthesis_gitbook-default}{%
\chapter{thesisdown::thesis\_gitbook: default}\label{thesisdownthesis_gitbook-default}}

Placeholder

\hypertarget{Bolum2}{%
\chapter{Değişim Noktası}\label{Bolum2}}

Placeholder

\hypertarget{tek-deux11fiux15fim-noktasux131-tespiti}{%
\section{Tek Değişim Noktası Tespiti}\label{tek-deux11fiux15fim-noktasux131-tespiti}}

\hypertarget{birden-fazla-deux11fiux15fim-noktasux131-tespiti}{%
\section{Birden Fazla Değişim Noktası Tespiti}\label{birden-fazla-deux11fiux15fim-noktasux131-tespiti}}

\hypertarget{ikili-segmentasyon-algoritmasux131}{%
\subsection{İkili Segmentasyon Algoritması}\label{ikili-segmentasyon-algoritmasux131}}

\hypertarget{prophet}{%
\subsection{PROPHET}\label{prophet}}

\hypertarget{peltpruned-exact-linear-time}{%
\subsection{PELT(Pruned Exact Linear Time)}\label{peltpruned-exact-linear-time}}

\hypertarget{optimal-buxf6luxfctleme}{%
\subsubsection{Optimal Bölütleme}\label{optimal-buxf6luxfctleme}}

\hypertarget{pelt-yuxf6ntemi}{%
\subsubsection{PELT Yöntemi}\label{pelt-yuxf6ntemi}}

\hypertarget{paruxe7alux131-regresyon}{%
\section{Parçalı Regresyon}\label{paruxe7alux131-regresyon}}

\hypertarget{kapsama-metriux11fi}{%
\section{KAPSAMA METRİĞİ}\label{kapsama-metriux11fi}}

\hypertarget{f1-puani}{%
\section{F1 PUANI}\label{f1-puani}}

\hypertarget{doux11fruluk-accuracy}{%
\subsubsection{Doğruluk (Accuracy)}\label{doux11fruluk-accuracy}}

\hypertarget{duyarlux131lux131k-recall}{%
\subsubsection{Duyarlılık (Recall)}\label{duyarlux131lux131k-recall}}

\hypertarget{kesinlik-precision}{%
\subsubsection{Kesinlik (Precision)}\label{kesinlik-precision}}

\hypertarget{f1-skoru-f1-score}{%
\subsubsection{F1 Skoru (F1 Score)}\label{f1-skoru-f1-score}}

\hypertarget{Bolum3}{%
\chapter{Uygulama}\label{Bolum3}}

\hypertarget{veri}{%
\section{Veri}\label{veri}}

\hypertarget{geruxe7ek-veri}{%
\subsection{Gerçek Veri}\label{geruxe7ek-veri}}

\begin{longtable}[]{@{}
  >{\raggedright\arraybackslash}p{(\columnwidth - 12\tabcolsep) * \real{0.1981}}
  >{\raggedright\arraybackslash}p{(\columnwidth - 12\tabcolsep) * \real{0.1415}}
  >{\raggedright\arraybackslash}p{(\columnwidth - 12\tabcolsep) * \real{0.1321}}
  >{\raggedright\arraybackslash}p{(\columnwidth - 12\tabcolsep) * \real{0.1321}}
  >{\raggedright\arraybackslash}p{(\columnwidth - 12\tabcolsep) * \real{0.1321}}
  >{\raggedright\arraybackslash}p{(\columnwidth - 12\tabcolsep) * \real{0.1321}}
  >{\raggedright\arraybackslash}p{(\columnwidth - 12\tabcolsep) * \real{0.1321}}@{}}
\caption{\label{tab:nvar1} Gerçek Veri}\tabularnewline
\toprule\noalign{}
\begin{minipage}[b]{\linewidth}\raggedright
Veri
\end{minipage} & \begin{minipage}[b]{\linewidth}\raggedright
Gözlem Sayısı
\end{minipage} & \begin{minipage}[b]{\linewidth}\raggedright
1. DN Sayısı
\end{minipage} & \begin{minipage}[b]{\linewidth}\raggedright
2. DN Sayısı
\end{minipage} & \begin{minipage}[b]{\linewidth}\raggedright
3. DN Sayısı
\end{minipage} & \begin{minipage}[b]{\linewidth}\raggedright
4. DN Sayısı
\end{minipage} & \begin{minipage}[b]{\linewidth}\raggedright
5. DN Sayısı
\end{minipage} \\
\midrule\noalign{}
\endfirsthead
\toprule\noalign{}
\begin{minipage}[b]{\linewidth}\raggedright
Veri
\end{minipage} & \begin{minipage}[b]{\linewidth}\raggedright
Gözlem Sayısı
\end{minipage} & \begin{minipage}[b]{\linewidth}\raggedright
1. DN Sayısı
\end{minipage} & \begin{minipage}[b]{\linewidth}\raggedright
2. DN Sayısı
\end{minipage} & \begin{minipage}[b]{\linewidth}\raggedright
3. DN Sayısı
\end{minipage} & \begin{minipage}[b]{\linewidth}\raggedright
4. DN Sayısı
\end{minipage} & \begin{minipage}[b]{\linewidth}\raggedright
5. DN Sayısı
\end{minipage} \\
\midrule\noalign{}
\endhead
\bottomrule\noalign{}
\endlastfoot
Bitcoin & NA & 4 & 1 & 1 & 7 & 7 \\
Brent-spot & 500 & 3 & 2 & 5 & 9 & 11 \\
children-per women & 301 & 2 & 1 & 2 & 4 & 2 \\
co2- canada & 215 & 2 & 6 & 2 & 5 & 7 \\
debt -Ireland & 21 & 2 & 2 & 2 & 4 & 2 \\
rail-lines & 37 & 2 & 2 & 2 & 2 & 1 \\
rather-stock & 600 & 1 & 2 & 1 & 2 & 2 \\
scanline-42049 & 481 & 10 & 7 & 2 & 7 & 7 \\
shangai-license & 205 & 1 & 1 & 1 & 1 & 2 \\
usd-isk & 247 & 2 & 4 & 1 & 3 & 2 \\
well-log & 675 & 11 & 9 & 9 & 2 & 17 \\
\end{longtable}

\hypertarget{yapay-veri}{%
\subsection{Yapay Veri}\label{yapay-veri}}

\begin{longtable}[]{@{}llll@{}}
\caption{\label{tab:nvar2} Yapay Veri}\tabularnewline
\toprule\noalign{}
Veri & Gözlem Sayısı & DN Sayısı & DN Konumu \\
\midrule\noalign{}
\endfirsthead
\toprule\noalign{}
Veri & Gözlem Sayısı & DN Sayısı & DN Konumu \\
\midrule\noalign{}
\endhead
\bottomrule\noalign{}
\endlastfoot
1 & 1687 & 4 & 516, 578, 779, 1499 \\
2 & 2092 & 3 & 564, 1003, 1347 \\
3 & 1582 & 4 & 175, 553, 1186, 1347 \\
4 & 2798 & 3 & 951, 985, 2315 \\
5 & 2165 & 3 & 1034, 1835, 1892 \\
6 & 1590 & 4 & 631, 698, 1208, 1481 \\
7 & 2244 & 1 & 1578 \\
8 & 2369 & 3 & 788, 958, 1768 \\
9 & 2288 & 4 & 316, 493, 587, 1606 \\
10 & 1847 & 4 & 153, 300, 469, 1172 \\
11 & 2756 & 3 & 2119, 2168, 2377 \\
12 & 2195 & 5 & 909, 1004, 1317, 1422, 1749 \\
13 & 1689 & 2 & 479, 611 \\
14 & 893 & 2 & 552, 837 \\
15 & 2562 & 1 & 575 \\
16 & 1978 & 1 & 293 \\
17 & 1992 & 2 & 955, 1798 \\
18 & 2472 & 3 & 1470, 1786, 2365 \\
19 & 2411 & 3 & 393, 874, 1047 \\
20 & 2297 & 2 & 79, 1622 \\
\end{longtable}

\#\#Gerçek Veriler

\hypertarget{f1-default}{%
\subsection{F1 Default}\label{f1-default}}

\begin{longtable}[]{@{}llllll@{}}
\caption{\label{tab:nvar3} F1 Default}\tabularnewline
\toprule\noalign{}
Veri & AMOC & BINSEG & PELT & SEGMENTED & PROPHET \\
\midrule\noalign{}
\endfirsthead
\toprule\noalign{}
Veri & AMOC & BINSEG & PELT & SEGMENTED & PROPHET \\
\midrule\noalign{}
\endhead
\bottomrule\noalign{}
\endlastfoot
Bitcoin & 0,3670 & 0,4897 & 0,2663 & 0,3670 & 0,2341 \\
Brent-spot & 0,2718 & 0,6431 & 0,4244 & 0,6590 & 0,3904 \\
children-per women & 0,6175 & 0,5902 & 0,3366 & 0,8440 & 0,5168 \\
co2- canada & 0,5441 & 0,8194 & 0,8194 & 0,3938 & 0,6315 \\
debt -Ireland & 0,7603 & 1,0000 & 0,7603 & 0,7603 & 0,6086 \\
rail-lines & 0,8462 & 0,8000 & 0,4690 & 1,0000 & 0,2666 \\
rather-stock & 0,2718 & 0,3392 & 0,4710 & 0,4886 & 0,5292 \\
scanline-42049 & 0,4926 & 0,7400 & 0,5151 & 0,2463 & 0,3902 \\
shangai-license & 0,8679 & 0,6511 & 0,6666 & 0,6495 & 0,5316 \\
usd-isk & 0,7854 & 0,6093 & NA & 0,7881 & 0,5956 \\
well-log & NA & 0,7289 & 0,4235 & 0,6912 & 0,3589 \\
\end{longtable}

Varsayılan parametre ayarlarıyla algoritmalar çalıştırıldığında, on bir verinin beşinde BinSeg, dördünde AMOC ve parçalı regresyon, ikisinde PELT 0,7 ve üzerinde F1 puanına sahipken Prophet algoritmasında F1 puanı 0,7 ve üzerinde olan veri bulunmamaktadır.

``Varsayılan ayarlarda, on bir verinin ikisinde AMOC ve Prophet ile, birinde PELT ve parçalı regresyon ile 0,3'ün altında F1 puanı elde edilmiştir.''\\
BinSeg algoritmasında F1 puanı 0,3 ve altında olan veri bulunmamaktadır.\\
Değişim noktası tespit edilemeyen durumlar (NA) ile gösterilmiştir.

\hypertarget{cover-default}{%
\subsection{Cover Default}\label{cover-default}}

\begin{longtable}[]{@{}llllll@{}}
\caption{\label{tab:nvar4} Cover Default}\tabularnewline
\toprule\noalign{}
Veri & AMOC & BINSEG & PELT & SEGMENTED & PROPHET \\
\midrule\noalign{}
\endfirsthead
\toprule\noalign{}
Veri & AMOC & BINSEG & PELT & SEGMENTED & PROPHET \\
\midrule\noalign{}
\endhead
\bottomrule\noalign{}
\endlastfoot
Bitcoin & 0,7640 & 0,7354 & 0,3022 & 0,5322 & 0,1941 \\
Brent-spot & 0,4251 & 0,5921 & 0,4535 & 0,4945 & 0,2992 \\
children-per women & 0,7838 & 0,7663 & 0,7721 & 0,6282 & 0,2753 \\
co2- canada & 0,5264 & 0,7291 & 0,7393 & 0,5135 & 0,3409 \\
debt -Ireland & 0,5844 & 0,6607 & 0,5446 & 0,5861 & 0,4000 \\
rail-lines & 0,7682 & 0,7732 & 0,4408 & 0,7890 & 0,3081 \\
rather-stock & 0,3870 & 0,3923 & 0,3970 & 0,5164 & 0,3470 \\
scanline-42049 & 0,4305 & 0,7502 & 0,4157 & 0,3859 & 0,3249 \\
shangai-license & 0,9105 & 0,7691 & 0,3132 & 0,8272 & 0,2458 \\
usd-isk & 0,8577 & NA & NA & 0,5248 & 0,2752 \\
well-log & 0,4527 & 0,7696 & 0,4285 & 0,4146 & 0,3743 \\
\end{longtable}

Varsayılan ayarlar ile, on bir verinin yedisinde BinSeg, beşinde AMOC, ikisinde PELT ve parçalı regresyon ile 0,7 ve üzerinde kapsama ölçütü değerleri elde edilmiştir.\\
Prophet algoritmasında hiçbir veri için kapsama ölçütü değeri 0,7 nin üzerinde değildir.\\
Varsayılan ayarlar ile Prophet algoritmasında başarılı kapsama ölçütü değerleri elde edilemediği görülmektedir. Prophet on bir verinin beşinde 0,3 ve altında kapsama ölçütü değerine sahiptir.\\
Değişim noktası tespit edilemeyen durumlar (NA) ile gösterilmiştir.

\hypertarget{f1-oracle}{%
\subsection{F1 Oracle}\label{f1-oracle}}

\begin{longtable}[]{@{}llllll@{}}
\caption{\label{tab:nvar5} F1 Oracle}\tabularnewline
\toprule\noalign{}
Veri & AMOC & BINSEG & PELT & SEGMENTED & PROPHET \\
\midrule\noalign{}
\endfirsthead
\toprule\noalign{}
Veri & AMOC & BINSEG & PELT & SEGMENTED & PROPHET \\
\midrule\noalign{}
\endhead
\bottomrule\noalign{}
\endlastfoot
Bitcoin & 0,3670 & 0,6124 & 0,3940 & 0,4523 & 0,2743 \\
Brent-spot & 0,2718 & 0,6341 & 0,4481 & 0,6590 & 0,3704 \\
children-per women & 0,6175 & 0,5902 & 0,6388 & 0,8440 & 0,5168 \\
co2- canada & 0,5441 & 0,8776 & 0,3595 & 0,7142 & 0,6045 \\
debt -Ireland & 0,7603 & 0,9583 & 0,9583 & 0,9795 & 0,9795 \\
rail-lines & 0,8461 & 0,8316 & 0,7234 & 0,9655 & 0,9655 \\
rather-stock & 0,2718 & 0,3728 & 0,5316 & 0,4243 & 0,5292 \\
scanline-42049 & 0,4926 & 0,8331 & 0,5704 & 0,8648 & 0,4581 \\
shangai-license & 0,8679 & 0,6511 & 0,2025 & 0,6495 & 0,5316 \\
usd-isk & 0,7854 & 0,6093 & 0,6007 & 0,7881 & 0,5956 \\
well-log & 0,2791 & 0,7289 & 0,5760 & 0,6912 & 0,3589 \\
\end{longtable}

Ayarlanmış parametreler ile en iyi F1 puanı değerleri parçalı regresyon ile alınmıştır. On bir veri setinin altısında parçalı regresyon, beşinde BinSeg, dördünde AMOC, ikisinde PELT ve Prophet'in 0,7 ve üzerinde F1 puanına sahip olduğu görülmektedir.

Tablo xx'e göre parçalı regresyon ve BinSeg algoritmaları denenen tüm gerçek veriler için 0,3 ve üzerinde F1 puanı elde etmişlerdir.Ancak AMOC algoritması için F1 puanı onbir verinin üçünde 0,3'ün altındadır.\\
\#\#\# Cover Oracle

\begin{longtable}[]{@{}llllll@{}}
\caption{\label{tab:nvar6} Cover Oracle}\tabularnewline
\toprule\noalign{}
Veri & AMOC & BINSEG & PELT & SEGMENTED & PROPHET \\
\midrule\noalign{}
\endfirsthead
\toprule\noalign{}
Veri & AMOC & BINSEG & PELT & SEGMENTED & PROPHET \\
\midrule\noalign{}
\endhead
\bottomrule\noalign{}
\endlastfoot
Bitcoin & 0,7640 & 0,4352 & 0,2182 & 0,3833 & 0,2045 \\
Brent-spot & 0,4251 & 0,4858 & 0,3257 & 0,5273 & 0,3206 \\
children-per women & 0,7938 & 0,7625 & 0,7635 & 0,7062 & 0,3485 \\
co2- canada & 0,5264 & 0,7001 & 0,7494 & 0,6622 & 0,4366 \\
debt -Ireland & 0,5844 & 0,8136 & 0,7042 & 0,8217 & 0,6919 \\
rail-lines & 0,7665 & 0,7731 & 0,5113 & 0,6964 & 0,6222 \\
rather-stock & 0,3870 & 0,3941 & 0,4840 & 0,4488 & 0,3470 \\
scanline-42049 & 0,4305 & 0,8570 & 0,4157 & 0,7944 & 0,4097 \\
shangai-license & 0,9105 & 0,7961 & 0,3955 & 0,8058 & 0,3625 \\
usd-isk & 0,8577 & 0,7549 & 0,5609 & 0,7819 & 0,4211 \\
well-log & 0,4527 & 0,6371 & 0,4873 & 0,5511 & 0,4615 \\
\end{longtable}

Ayarlanmış parametreler ile algoritmalar çalıştırıldığında, kapsama ölçütüne göre en iyi sonuçları BinSeg algoritması vermiştir. Denenen on bir verinin yedisinde BinSeg, beşinde AMOC ve parçalı regresyon, üçünde PELT 0,7 ve üzerinde kapsama metriğine sahipken Prophet algoritmasında kapsama metriği 0,7 ve üzerinde olan veri bulunmamaktadır.

Çalışmada incelenen algoritmaların ihtiyaç duyduğu parametre değerleri verilere göre en uygun hale getirildiğinde BinSeg ve parçalı regresyon algoritmalarının diğerlerine göre değişim noktalarının konumlarını belirlemede daha başarılı olduğu görülmüştür.

\hypertarget{yapay-veriler}{%
\section{Yapay Veriler}\label{yapay-veriler}}

\hypertarget{f1-default-1}{%
\subsection{F1 Default}\label{f1-default-1}}

\begin{longtable}[]{@{}llllll@{}}
\caption{\label{tab:nvar7} F1 Default}\tabularnewline
\toprule\noalign{}
Veri & AMOC & BINSEG & SEGMENTED & PROPHET & PELT \\
\midrule\noalign{}
\endfirsthead
\toprule\noalign{}
Veri & AMOC & BINSEG & SEGMENTED & PROPHET & PELT \\
\midrule\noalign{}
\endhead
\bottomrule\noalign{}
\endlastfoot
1 & 0,5714 & 0,9091 & 0,2857 & 0,2857 & 0,4000 \\
2 & 0,6667 & 0,4000 & 0,3333 & 0,1818 & 0,4444 \\
3 & 0,5714 & 0,9091 & 0,1290 & 0,2000 & 0,4000 \\
4 & 0,6667 & 0,8000 & 0,3333 & 0,2222 & 0,7500 \\
5 & 0,6667 & 0,6000 & 0,4444 & 0,2500 & 0,6666 \\
6 & 0,5714 & 0,5455 & 0,4000 & 0,1538 & 0,4000 \\
7 & 0,5000 & 0,2500 & 0,5000 & 0,3333 & 0,5000 \\
8 & 0,6667 & 0,8000 & 0,5000 & 0,2222 & 0,6000 \\
9 & 0,2857 & 0,5455 & 0,4000 & 0,1818 & 0,3636 \\
10 & 0,5714 & 0,7273 & 0,6000 & 0,1666 & 0,4615 \\
11 & 0,6667 & 0,6000 & 0,5000 & 0,2222 & 0,4444 \\
12 & 0,5000 & 0,8333 & 0,1666 & 0,1666 & 0,5000 \\
13 & 0,8000 & 0,6667 & 0,6666 & 0,2857 & 0,5714 \\
14 & 0,8000 & 0,6667 & 0,6666 & 0,2857 & 0,5714 \\
15 & 0,9998 & 0,5000 & 0,5000 & 0,4000 & 0,4000 \\
16 & 0,9999 & 0,5000 & 0,5000 & 0,4000 & 0,4000 \\
17 & 0,8000 & 0,6667 & 0,6666 & 0,2857 & 0,5714 \\
18 & 0,6667 & 0,8000 & 0,5000 & 0,2222 & 0,4000 \\
19 & 0,6667 & 0,8000 & 0,5000 & 0,2222 & 0,4444 \\
20 & 0,8000 & 0,4444 & 0,3333 & 0,2857 & 0,2857 \\
\end{longtable}

Algoritmalar varsayılan parametre ayarlarıyla çalıştırıldığında, yirmi simülasyon verisinin sekizinde BinSeg, altısında AMOC ve birinde PELT 0,7 ve üzerinde F1 puanına sahipken\\
parçalı regresyon ve Prophet algoritmasında F1 puanı 0,7 ve üzerinde olan veri bulunmamaktadır.\\
Varsayılan parametrelerle çalıştırılan algoritmalardan Prophet on yedi, parçalı regresyon üç, AMOC, BinSeg ve Pelt algoritmalarında birer tane 0,3'ün altında F1 puanına sahip veri bulunmaktadır.

\hypertarget{cover-default-1}{%
\subsection{COVER Default}\label{cover-default-1}}

\begin{longtable}[]{@{}llllll@{}}
\caption{\label{tab:nvar8} Cover Default}\tabularnewline
\toprule\noalign{}
Veri & AMOC & BINSEG & SEGMENTED & PROPHET & PELT \\
\midrule\noalign{}
\endfirsthead
\toprule\noalign{}
Veri & AMOC & BINSEG & SEGMENTED & PROPHET & PELT \\
\midrule\noalign{}
\endhead
\bottomrule\noalign{}
\endlastfoot
1 & 0,5975 & 0,9895 & 0,3740 & 0,3607 & 0,6843 \\
2 & 0,5388 & 0,8881 & 0,5185 & 0,5600 & 0,7686 \\
3 & 0,4937 & 0,9943 & 0,2397 & 0,5151 & 0,7953 \\
4 & 0,5850 & 0,8989 & 0,4876 & 0,5495 & 0,8897 \\
5 & 0,7707 & 0,9687 & 0,8126 & 0,6164 & 0,7706 \\
6 & 0,6079 & 0,8822 & 0,7007 & 0,3625 & 0,5339 \\
7 & 0,9648 & 0,7995 & 0,7807 & 0,4670 & 0,8523 \\
8 & 0,6084 & 0,8848 & 0,6725 & 0,6173 & 0,7047 \\
9 & 0,6629 & 0,9467 & 0,4937 & 0,5129 & 0,6060 \\
10 & 0,6269 & 0,9288 & 0,7368 & 0,4726 & 0,4510 \\
11 & 0,8764 & 0,9464 & 0,8185 & 0,9991 & 0,6856 \\
12 & 0,5639 & 0,8220 & 0,6032 & 0,6158 & 0,5461 \\
13 & 0,8771 & 0,7478 & 0,8771 & 0,5563 & 0,4660 \\
14 & 0,8932 & 0,9567 & 0,8911 & 0,4304 & 0,7085 \\
15 & 0,9992 & 0,9863 & 0,8634 & 0,5091 & 0,4972 \\
16 & 0,9950 & 0,7715 & 0,8828 & 0,4541 & 0,4058 \\
17 & 0,8408 & 0,8620 & 0,8403 & 0,5809 & 0,5871 \\
18 & 0,7744 & 0,9956 & 0,8502 & 0,4184 & 0,4871 \\
19 & 0,6932 & 0,9925 & 0,7623 & 0,4422 & 0,4285 \\
20 & 0,5905 & 0,7208 & 0,4578 & 0,4711 & 0,5644 \\
\end{longtable}

BinSeg varsayılan parametre ayarlarıyla çalıştırıldığında tüm verilerde 0,7'nin üzerinde kapsama ölçütü değerine sahiptir. Parçalı regresyon on ikisinde, AMOC dokuzunda, Pelt yedisinde ve Prophet birinde 0,7 ve üzerinde kapsama ölçütü değerine sahip olduğu görülmektedir.\\
Varsayılan ayarlar çalıştırılan algoritmalardan sadece parçalı regresyonda bir veri 0,3'ün altında kapsama ölçütü değerine sahiptir.

\hypertarget{f1-oracle-1}{%
\subsection{F1 Oracle}\label{f1-oracle-1}}

\begin{longtable}[]{@{}llllll@{}}
\caption{\label{tab:nvar9} F1 Oracle}\tabularnewline
\toprule\noalign{}
Veri & AMOC & BINSEG & SEGMENTED & PROPHET & PELT \\
\midrule\noalign{}
\endfirsthead
\toprule\noalign{}
Veri & AMOC & BINSEG & SEGMENTED & PROPHET & PELT \\
\midrule\noalign{}
\endhead
\bottomrule\noalign{}
\endlastfoot
1 & 0,5714 & 0,9997 & 0,4444 & 0,2500 & 0,6000 \\
2 & 0,6667 & 0,5000 & 0,5000 & 0,4000 & 0,5000 \\
3 & 0,5714 & 0,9994 & 0,4000 & 0,2500 & 0,4000 \\
4 & 0,6667 & 0,9996 & 0,4000 & 0,2500 & 0,7500 \\
5 & 0,6667 & 0,7500 & 0,4000 & 0,2857 & 0,6666 \\
6 & 0,5714 & 0,6000 & 0,4000 & 0,3333 & 0,2857 \\
7 & 0,5000 & 0,5000 & 0,5000 & 0,4000 & 0,5000 \\
8 & 0,6667 & 0,7500 & 0,5000 & 0,2500 & 0,5454 \\
9 & 0,2857 & 0,6000 & 0,4000 & 0,2222 & 0,3636 \\
10 & 0,5714 & 0,8000 & 0,6000 & 0,2000 & 0,6000 \\
11 & 0,6667 & 0,7500 & 0,5000 & 0,2500 & 0,5714 \\
12 & 0,5000 & 0,8333 & 0,1666 & 0,3636 & 0,5000 \\
13 & 0,8000 & 0,9987 & 0,6666 & 0,3333 & 0,6666 \\
14 & 0,8000 & 0,6667 & 0,6666 & 0,3333 & 0,6666 \\
15 & 0,9998 & 0,9989 & 0,5000 & 0,5000 & 0,5000 \\
16 & 0,9999 & 0,9967 & 0,5000 & 0,5000 & 0,5000 \\
17 & 0,8000 & 0,9985 & 0,6666 & 0,3333 & 0,6666 \\
18 & 0,6667 & 0,9978 & 0,5000 & 0,2500 & 0,5000 \\
19 & 0,6667 & 0,9949 & 0,5000 & 0,2500 & 0,5000 \\
20 & 0,8000 & 0,6667 & 0,3333 & 0,3333 & 0,3333 \\
\end{longtable}

Ayarlanmış parametreler sonucunda en iyi F1 puanı değerleri BinSeg ile elde edilmiştir. Prophet ve parçalı regresyon algoritmalarında hiç bir veri 0,7 ve üzerinde F1 puanına sahip değildir. Pelt algoritmasında bir tane verinin 0,7'nin üzerinde olduğu görülmektedir,\\
Prophet algoritmasının en kötü F1 puanı değerlerini verdiği görülmektedir. AMOC, parçalı regresyon ve Pelt algoritmalarında 0,3 'ün altında birer tane veri bulunmaktadır.

\hypertarget{cover-oracle}{%
\subsection{COVER Oracle}\label{cover-oracle}}

\begin{longtable}[]{@{}llllll@{}}
\caption{\label{tab:nvar10} Cover Oracle}\tabularnewline
\toprule\noalign{}
Veri & AMOC & BINSEG & SEGMENTED & PROPHET & PELT \\
\midrule\noalign{}
\endfirsthead
\toprule\noalign{}
Veri & AMOC & BINSEG & SEGMENTED & PROPHET & PELT \\
\midrule\noalign{}
\endhead
\bottomrule\noalign{}
\endlastfoot
1 & 0,5975 & 0,9930 & 0,9930 & 0,6104 & 0,7081 \\
2 & 0,5388 & 0,8914 & 0,8914 & 0,6793 & 0,8194 \\
3 & 0,4937 & 0,9950 & 0,9950 & 0,6152 & 0,9246 \\
4 & 0,5850 & 0,9957 & 0,9957 & 0,6830 & 0,8840 \\
5 & 0,7707 & 0,9688 & 0,9688 & 0,7494 & 0,7706 \\
6 & 0,6079 & 0,8853 & 0,8853 & 0,3967 & 0,5599 \\
7 & 0,9648 & 0,9648 & 0,9648 & 0,6002 & 0,9670 \\
8 & 0,6084 & 0,8831 & 0,8831 & 0,6700 & 0,7364 \\
9 & 0,6629 & 0,9467 & 0,9467 & 0,5665 & 0,6067 \\
10 & 0,6269 & 0,9092 & 0,9092 & 0,4866 & 0,7201 \\
11 & 0,8764 & 0,9689 & 0,9689 & 0,9998 & 0,9387 \\
12 & 0,5639 & 0,8220 & 0,8220 & 0,6338 & 0,5604 \\
13 & 0,8771 & 0,9976 & 0,9976 & 0,6158 & 0,6509 \\
14 & 0,8932 & 0,9567 & 0,9567 & 0,5356 & 0,8589 \\
15 & 0,9992 & 0,9992 & 0,9992 & 0,5255 & 0,8189 \\
16 & 0,9950 & 0,9950 & 0,9950 & 0,5825 & 0,7296 \\
17 & 0,8408 & 0,9980 & 0,9980 & 0,7171 & 0,7905 \\
18 & 0,7744 & 0,9976 & 0,9976 & 0,4701 & 0,6171 \\
19 & 0,6932 & 0,9983 & 0,9983 & 0,4816 & 0,6680 \\
20 & 0,5905 & 0,9854 & 0,9854 & 0,5511 & 0,7184 \\
\end{longtable}

Ayarlanan parametreler ile hesaplanan kapsama ölçütü değerinin BinSeg ve parçalı regresyon algoritmalarının tamamında 0,8'in üzerinde olduğu görülmektedir. Pelt için on dört, AMOC için dokuz ve Prophet algortiması için üç verinin 0,7'nin üzerinde\\
kapsama ölçütü değerine sahip olduğu görülmektedir.\\
Ayarlanan parametrelerle çalıştırılan beş farklı algoritmanın kullanıldığı yirmi verinin hiçbirinde 0,3'ün altından kapsama ölçütü değeri bulunmamaktadır.

\hypertarget{sporcu-verileri}{%
\section{Sporcu Verileri}\label{sporcu-verileri}}

\hypertarget{cover-oracle-1}{%
\subsection{Cover Oracle}\label{cover-oracle-1}}

\begin{longtable}[]{@{}llll@{}}
\caption{\label{tab:nvar11} Cover Oracle}\tabularnewline
\toprule\noalign{}
Veri & BINSEG & SEGMENTED & PELT \\
\midrule\noalign{}
\endfirsthead
\toprule\noalign{}
Veri & BINSEG & SEGMENTED & PELT \\
\midrule\noalign{}
\endhead
\bottomrule\noalign{}
\endlastfoot
1 & 0,7023 & 0,6688 & 0,7340 \\
2 & NA & 0,8232 & NA \\
3 & NA & 0,5097 & NA \\
4 & 0,7485 & 0,9547 & 0,7308 \\
5 & 0,9326 & 0,8972 & 0,7314 \\
6 & NA & 0,7793 & NA \\
7 & NA & 0,9493 & NA \\
8 & 1,0000 & 0,8930 & 0,8425 \\
9 & 0,8181 & 0,9545 & 0,8723 \\
10 & 0,8039 & 0,6843 & 0,7189 \\
11 & 1,0000 & 0,6111 & 0,5965 \\
12 & 1,0000 & 0,7270 & 0,9353 \\
13 & 0,6520 & 0,8486 & 0,9805 \\
14 & 0,5434 & 0,7423 & 1,0000 \\
15 & 0,7141 & 0,4571 & 0,8194 \\
\end{longtable}

Ayarlanan parametreler ile hesaplanan kapsama ölçütü değerinin parçalı regresyon algoritmasında yedi, Pelt ve BinSeg algoritmalarında ise altı veride 0,8'in üzerinde olduğu görülmektedir.\\
Ayarlanan parametrelerle çalıştırılan üç farklı algoritmanın kullanıldığı on beş verinin hiçbirinde 0,3'ün altından kapsama ölçütü değeri bulunmamaktadır.\\
İki değişim noktası tespit edilemeyen durumlar (NA) ile gösterilmiştir.

\hypertarget{f1-oracle-2}{%
\subsection{F1 Oracle}\label{f1-oracle-2}}

\begin{longtable}[]{@{}llll@{}}
\caption{\label{tab:nvar12} F1 Oracle}\tabularnewline
\toprule\noalign{}
Veri & BINSEG & SEGMENTED & PELT \\
\midrule\noalign{}
\endfirsthead
\toprule\noalign{}
Veri & BINSEG & SEGMENTED & PELT \\
\midrule\noalign{}
\endhead
\bottomrule\noalign{}
\endlastfoot
1 & 0,6600 & 0,6600 & 0,6600 \\
2 & NA & 1,0000 & NA \\
3 & NA & 0,6600 & NA \\
4 & 1,0000 & 1,0000 & 1,0000 \\
5 & 1,0000 & 1,0000 & 1,0000 \\
6 & NA & 1,0000 & NA \\
7 & NA & 1,0000 & NA \\
8 & 1,0000 & 1,0000 & 1,0000 \\
9 & 0,8571 & 0,8571 & 1,0000 \\
10 & 1,0000 & 1,0000 & 1,0000 \\
11 & 0,7499 & 0,8571 & 0,8571 \\
12 & 1,0000 & 1,0000 & 1,0000 \\
13 & 0,5714 & 0,8571 & 1,0000 \\
14 & 0,2857 & 0,8571 & 1,0000 \\
15 & 0,2857 & 0,4615 & 0,6600 \\
\end{longtable}

Ayarlanmış parametreler sonucunda en iyi F1 puanı değerleri parçalı regresyon ile elde edilmiştir. Parçalı regresyon algoritmasında on iki veri 0,8 ve üzerinde F1 puanına sahiptir. Pelt algoritmasında dokuz, BinSeg algoritmasında ise altı verinin 0,8'in üzerinde F1 puanına sahip olduğu görülmektedir.\\
Pelt ve parçalı regresyon algoritmalarının 0,3'ün altında F1 puanı yokken, BinSeg algoritmasında ise iki veride 0,3'ün altında F1 puanı elde edilmiştir.\\
İki değişim noktası tespit edilemeyen durumlar (NA) ile gösterilmiştir.\\
\#\# Çalışma Tasarımı

\hypertarget{sonuuxe7lar}{%
\section{Sonuçlar}\label{sonuuxe7lar}}

\hypertarget{Bolum4}{%
\chapter{Bölüm Başlığı}\label{Bolum4}}

\hypertarget{bu-bir-alt-baux15flux131k}{%
\section{Bu bir alt başlık}\label{bu-bir-alt-baux15flux131k}}

Bu bölümde şu konular yer almaktadır\ldots{}

\hypertarget{bu-ikinci-seviye-bir-alt-baux15flux131k}{%
\subsection{Bu ikinci seviye bir alt başlık}\label{bu-ikinci-seviye-bir-alt-baux15flux131k}}

\hypertarget{Bolum5}{%
\chapter{Bölüm 4 Başlık}\label{Bolum5}}

\hypertarget{bu-bir-alt-baux15flux131k-1}{%
\section{Bu bir alt başlık}\label{bu-bir-alt-baux15flux131k-1}}

Bu bölümde şu konular yer almaktadır\ldots{}

\hypertarget{bu-ikinci-seviye-bir-alt-baux15flux131k-1}{%
\subsection{Bu ikinci seviye bir alt başlık}\label{bu-ikinci-seviye-bir-alt-baux15flux131k-1}}

\hypertarget{sonuuxe7}{%
\chapter*{Sonuç}\label{sonuuxe7}}
\addcontentsline{toc}{chapter}{Sonuç}

If we don't want Conclusion to have a chapter number next to it, we can add the \texttt{\{-\}} attribute.

\textbf{More info}

And here's some other random info: the first paragraph after a chapter title or section head \emph{shouldn't be} indented, because indents are to tell the reader that you're starting a new paragraph. Since that's obvious after a chapter or section title, proper typesetting doesn't add an indent there.

\hypertarget{kaynaklar}{%
\chapter*{Kaynaklar}\label{kaynaklar}}
\addcontentsline{toc}{chapter}{Kaynaklar}

Placeholder

\appendix

\hypertarget{ilk-ek-baux15flux131ux11fux131}{%
\chapter{İlk Ek Başlığı}\label{ilk-ek-baux15flux131ux11fux131}}

This first appendix includes all of the R chunks of code that were hidden throughout the document (using the \texttt{include\ =\ FALSE} chunk tag) to help with readibility and/or setup.

\textbf{In the main Rmd file}

\textbf{In Chapter \ref{ref-labels}:}

\hypertarget{ikinci-ek-baux15flux131ux11fux131}{%
\chapter{İkinci Ek Başlığı}\label{ikinci-ek-baux15flux131ux11fux131}}

İkinci Ek




\end{document}
