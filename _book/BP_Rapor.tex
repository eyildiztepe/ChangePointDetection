% This template is borrowed from the Reed College LaTeX thesis template. Most of the work
% for the document class was done by Sam Noble (SN), as well as this
% template. Later comments etc. by Ben Salzberg (BTS). Additional
% restructuring and APA support by Jess Youngberg (JY).
% Your comments and suggestions are more than welcome; please email
% them to cus@reed.edu
%
% See http://web.reed.edu/cis/help/latex.html for help. There are a
% great bunch of help pages there, with notes on
% getting started, bibtex, etc. Go there and read it if you're not
% already familiar with LaTeX.
%
% Any line that starts with a percent symbol is a comment.
% They won't show up in the document, and are useful for notes
% to yourself and explaining commands.
% Commenting also removes a line from the document;
% very handy for troubleshooting problems. -BTS

% As far as I know, this follows the requirements laid out in
% the 2002-2003 Senior Handbook. Ask a librarian to check the
% document before binding. -SN

%%
%% Preamble
%%
% \documentclass{<something>} must begin each LaTeX document
\documentclass[12pt,twoside]{deuthesis}
% Packages are extensions to the basic LaTeX functions. Whatever you
% want to typeset, there is probably a package out there for it.
% Chemistry (chemtex), screenplays, you name it.
% Check out CTAN to see: http://www.ctan.org/
%%
\usepackage{graphicx,latexsym}
\usepackage{amsmath}
\usepackage{amssymb,amsthm}
\usepackage{longtable,booktabs,setspace}
\usepackage{chemarr} %% Useful for one reaction arrow, useless if you're not a chem major
\usepackage[hyphens]{url}
% Added by CII
\usepackage{hyperref}
\usepackage{lmodern}
\usepackage{float}
\floatplacement{figure}{H}
% End of CII addition
\usepackage{rotating}

% Next line commented out by CII
%%% \usepackage{natbib}
% Comment out the natbib line above and uncomment the following two lines to use the new
% biblatex-chicago style, for Chicago A. Also make some changes at the end where the
% bibliography is included.
%\usepackage{biblatex-chicago}
%\bibliography{thesis}


% Added by CII (Thanks, Hadley!)
% Use ref for internal links
\renewcommand{\hyperref}[2][???]{\autoref{#1}}
\def\chapterautorefname{Chapter}
\def\sectionautorefname{Section}
\def\subsectionautorefname{Subsection}
% End of CII addition

% Added by CII
\usepackage{caption}
\captionsetup{width=5in}
% End of CII addition

% \usepackage{times} % other fonts are available like times, bookman, charter, palatino

% Syntax highlighting #22

% To pass between YAML and LaTeX the dollar signs are added by CII
\title{Değişim Noktası Belirleme Yöntemleri ve Uygulamaları}
%\author{Pelin PEKERMerve AKEdanur Binnaz DURSUNAhmet ÇALI} %Tek yazar için
\author{Pelin PEKER \\ Merve AK \\ Edanur Binnaz DURSUN \\ Ahmet ÇALI} %Çok yazar için
% The month and year that you submit your FINAL draft TO THE LIBRARY (May or December)
\date{May 2024}
\division{İSTATİSTİK BÖLÜMÜ}
\advisor{Dr.~Engin YILDIZTEPE}
\institution{FEN FAKÜLTESİ}
\degree{Bitirme Projesi Raporu}
%If you have two advisors for some reason, you can use the following
% Uncommented out by CII
% End of CII addition

%%% Remember to use the correct department!
\department{İstatistik Bölümü}
% if you're writing a thesis in an interdisciplinary major,
% uncomment the line below and change the text as appropriate.
% check the Senior Handbook if unsure.
%\thedivisionof{The Established Interdisciplinary Committee for}
% if you want the approval page to say "Approved for the Committee",
% uncomment the next line
%\approvedforthe{Committee}

% Added by CII
%%% Copied from knitr
%% maxwidth is the original width if it's less than linewidth
%% otherwise use linewidth (to make sure the graphics do not exceed the margin)
\makeatletter
\def\maxwidth{ %
  \ifdim\Gin@nat@width>\linewidth
    \linewidth
  \else
    \Gin@nat@width
  \fi
}
\makeatother

\renewcommand{\contentsname}{Table of Contents}
% End of CII addition

\setlength{\parskip}{0pt}

% Added by CII

\providecommand{\tightlist}{%
  \setlength{\itemsep}{0pt}\setlength{\parskip}{0pt}}

\Acknowledgements{
Tüm çalışma süresince yönlendiriciliği, katkıları ve yardımları ile yanımızda olan danışmanımız Dr.~Engin YILDIZTEPE 'ye ve böyle bir çalışmayı yapmamız için bize fırsat tanıyan Dokuz Eylül Üniversitesi Fen Fakültesi İstatistik Bölümüne teşekkür ederiz.\\
\strut \\
\strut \\
Pelin PEKER\\
Merve AK\\
Edanur Binnaz DURSUN\\
Ahmet ÇALI\\
}

\Dedication{

}

\Preface{
``Değişim Noktası Belirleme Yöntemleri ve Uygulamaları'' başlıklı bitirme projesi raporu tarafıgitmdan okunmuş, kapsamı ve niteliği açısından bir Bitirme Projesi raporu olarak kabul edilmiştir.\\
\strut \\
\strut \\
Dr.~Engin YILDIZTEPE
}

\AbstractTR{
Değişim noktası verilerde meydana gelen beklenmedik değişiklikler olarak tanımlanabilir. Değişim noktası tespit yöntemleri bu noktaları istatistiksel tekniklerle bulmayı amaçlar. Değişim noktası analizi finans, kalite kontrol, ağ analizi gibi çok farklı alanlarda kullanılmaktadır. Bu çalışmada değişim noktası tespit yöntemleri incelenmiştir. Çalışma kapsamında AMOC, BinSeg, Parçalı Regresyon, Pelt ve Prophet algoritmaları kullanılmıştır. Algoritmaların uygulamadaki performanslarını belirlemek amacıyla yirmi yapay ve on bir gerçek veri kullanılmıştır. Algoritmaların performansları F1 puanı ve kapsama ölçütü kullanılarak değerlendirilmiştir. Değişim noktası içeren yapay veri üretmek ve bahsedilen algoritmaları uygulayabilmek amacıyla bir RShiny web uygulaması geliştirilmiştir. Çalışmada R ve Python programlama dilleri kullanılmıştır.\\
\strut \\

\textbf{Anahtar Kelimeler:} Değişim noktası, AMOC, BinSeg, Pelt, Prophet, Parçalı Regresyon, R Shiny
}

\Abstract{
Changepoints can be defined as unexpected alterations in data. Changepoint detection methods aim to identify these points using statistical techniques. Changepoint analysis is utilized in various fields such as finance, quality control, and network analysis. In this study, changepoint detection methods were examined. Within the scope of the study, AMOC, BinSeg,Segmented Regression, Pelt, and Prophet algorithms were employed. Twenty artificial and eleven real datasets were used to determine the performance of the algorithms. The performance of the algorithms was evaluated using the F1 score and cover metric. To generate artificial data containing changepoints and to apply the mentioned algorithms, an RShiny web application was developed. R and Python programming languages were used in the study.\\
\strut \\

\textbf{Keywords:} Changepoint, AMOC, BinSeg, Pelt, Prophet, Segmented Regression, R Shiny
}


% End of CII addition
%%
%% End Preamble
%%
%

\begin{document}

% Everything below added by CII
  \maketitle

\frontmatter % this stuff will be roman-numbered
\pagestyle{empty} % this removes page numbers from the frontmatter

\begin{preface}
	``Değişim Noktası Belirleme Yöntemleri ve Uygulamaları'' başlıklı bitirme projesi raporu tarafıgitmdan okunmuş, kapsamı ve niteliği açısından bir Bitirme Projesi raporu olarak kabul edilmiştir.\\
\strut \\
\strut \\
Dr.~Engin YILDIZTEPE
\end{preface}

  \begin{acknowledgements}
    Tüm çalışma süresince yönlendiriciliği, katkıları ve yardımları ile yanımızda olan danışmanımız Dr.~Engin YILDIZTEPE 'ye ve böyle bir çalışmayı yapmamız için bize fırsat tanıyan Dokuz Eylül Üniversitesi Fen Fakültesi İstatistik Bölümüne teşekkür ederiz.\\
    \strut \\
    \strut \\
    Pelin PEKER\\
    Merve AK\\
    Edanur Binnaz DURSUN\\
    Ahmet ÇALI\\
  \end{acknowledgements}

\begin{abstractTR}
	Değişim noktası verilerde meydana gelen beklenmedik değişiklikler olarak tanımlanabilir. Değişim noktası tespit yöntemleri bu noktaları istatistiksel tekniklerle bulmayı amaçlar. Değişim noktası analizi finans, kalite kontrol, ağ analizi gibi çok farklı alanlarda kullanılmaktadır. Bu çalışmada değişim noktası tespit yöntemleri incelenmiştir. Çalışma kapsamında AMOC, BinSeg, Parçalı Regresyon, Pelt ve Prophet algoritmaları kullanılmıştır. Algoritmaların uygulamadaki performanslarını belirlemek amacıyla yirmi yapay ve on bir gerçek veri kullanılmıştır. Algoritmaların performansları F1 puanı ve kapsama ölçütü kullanılarak değerlendirilmiştir. Değişim noktası içeren yapay veri üretmek ve bahsedilen algoritmaları uygulayabilmek amacıyla bir RShiny web uygulaması geliştirilmiştir. Çalışmada R ve Python programlama dilleri kullanılmıştır.\\
\strut \\

\textbf{Anahtar Kelimeler:} Değişim noktası, AMOC, BinSeg, Pelt, Prophet, Parçalı Regresyon, R Shiny
\end{abstractTR}

\begin{abstract}
	Changepoints can be defined as unexpected alterations in data. Changepoint detection methods aim to identify these points using statistical techniques. Changepoint analysis is utilized in various fields such as finance, quality control, and network analysis. In this study, changepoint detection methods were examined. Within the scope of the study, AMOC, BinSeg,Segmented Regression, Pelt, and Prophet algorithms were employed. Twenty artificial and eleven real datasets were used to determine the performance of the algorithms. The performance of the algorithms was evaluated using the F1 score and cover metric. To generate artificial data containing changepoints and to apply the mentioned algorithms, an RShiny web application was developed. R and Python programming languages were used in the study.\\
\strut \\

\textbf{Keywords:} Changepoint, AMOC, BinSeg, Pelt, Prophet, Segmented Regression, R Shiny
\end{abstract}


  \hypersetup{linkcolor=black}
  \setcounter{tocdepth}{2}
  \tableofcontents

  \listoftables

  \listoffigures


% This was added by EY

\mainmatter % here the regular arabic numbering starts
\pagestyle{fancyplain} % turns page numbering back on

\hypertarget{thesisdownthesis_gitbook-default}{%
\chapter{thesisdown::thesis\_gitbook: default}\label{thesisdownthesis_gitbook-default}}

Placeholder

\hypertarget{Bolum2}{%
\chapter{Değişim Noktası}\label{Bolum2}}

Placeholder

\hypertarget{tek-deux11fiux15fim-noktasux131-tespiti}{%
\section{Tek Değişim Noktası Tespiti}\label{tek-deux11fiux15fim-noktasux131-tespiti}}

\hypertarget{birden-fazla-deux11fiux15fim-noktasux131-tespiti}{%
\section{Birden Fazla Değişim Noktası Tespiti}\label{birden-fazla-deux11fiux15fim-noktasux131-tespiti}}

\hypertarget{ikili-segmentasyon-algoritmasux131}{%
\subsection{İkili Segmentasyon Algoritması}\label{ikili-segmentasyon-algoritmasux131}}

\hypertarget{prophet}{%
\subsection{PROPHET}\label{prophet}}

\hypertarget{peltpruned-exact-linear-time}{%
\subsection{PELT(Pruned Exact Linear Time)}\label{peltpruned-exact-linear-time}}

\hypertarget{optimal-buxf6luxfctleme}{%
\subsubsection{Optimal Bölütleme}\label{optimal-buxf6luxfctleme}}

\hypertarget{pelt-yuxf6ntemi}{%
\subsubsection{PELT Yöntemi}\label{pelt-yuxf6ntemi}}

\hypertarget{paruxe7alux131-regresyon}{%
\section{Parçalı Regresyon}\label{paruxe7alux131-regresyon}}

\hypertarget{kapsama-metriux11fi}{%
\section{KAPSAMA METRİĞİ}\label{kapsama-metriux11fi}}

\hypertarget{f1-puani}{%
\section{F1 PUANI}\label{f1-puani}}

\hypertarget{doux11fruluk-accuracy}{%
\subsubsection{Doğruluk (Accuracy)}\label{doux11fruluk-accuracy}}

\hypertarget{duyarlux131lux131k-recall}{%
\subsubsection{Duyarlılık (Recall)}\label{duyarlux131lux131k-recall}}

\hypertarget{kesinlik-precision}{%
\subsubsection{Kesinlik (Precision)}\label{kesinlik-precision}}

\hypertarget{f1-skoru-f1-score}{%
\subsubsection{F1 Skoru (F1 Score)}\label{f1-skoru-f1-score}}

\hypertarget{Bolum3}{%
\chapter{Uygulama}\label{Bolum3}}

Placeholder

\hypertarget{veri}{%
\section{Veri}\label{veri}}

\hypertarget{geruxe7ek-veri}{%
\subsection{Gerçek Veri}\label{geruxe7ek-veri}}

\hypertarget{yapay-veri}{%
\subsection{Yapay Veri}\label{yapay-veri}}

\hypertarget{geruxe7ek-veri-1}{%
\section{Gerçek Veri}\label{geruxe7ek-veri-1}}

\hypertarget{f1-default}{%
\subsection{F1 Default}\label{f1-default}}

\hypertarget{cover-default}{%
\subsection{Cover Default}\label{cover-default}}

\hypertarget{f1-oracle}{%
\subsection{F1 Oracle}\label{f1-oracle}}

\hypertarget{cover-oracle}{%
\subsection{Cover Oracle}\label{cover-oracle}}

\hypertarget{sporcu-verileri}{%
\section{Sporcu Verileri}\label{sporcu-verileri}}

\hypertarget{cover-oracle-1}{%
\subsection{Cover Oracle}\label{cover-oracle-1}}

\hypertarget{f1-oracle-1}{%
\subsection{F1 Oracle}\label{f1-oracle-1}}

\hypertarget{yapay-veri-1}{%
\section{Yapay Veri}\label{yapay-veri-1}}

\hypertarget{f1-default-1}{%
\subsection{F1 Default}\label{f1-default-1}}

\hypertarget{cover-default-1}{%
\subsection{Cover Default}\label{cover-default-1}}

\hypertarget{f1-oracle-2}{%
\subsection{F1 Oracle}\label{f1-oracle-2}}

\hypertarget{cover-oracle-2}{%
\subsection{Cover Oracle}\label{cover-oracle-2}}

\hypertarget{uxe7alux131ux15fma-tasarux131mux131}{%
\section{Çalışma Tasarımı}\label{uxe7alux131ux15fma-tasarux131mux131}}

\hypertarget{sonuuxe7lar}{%
\section{Sonuçlar}\label{sonuuxe7lar}}

\hypertarget{Bolum4}{%
\chapter{Bölüm Başlığı}\label{Bolum4}}

\hypertarget{bu-bir-alt-baux15flux131k}{%
\section{Bu bir alt başlık}\label{bu-bir-alt-baux15flux131k}}

Bu bölümde şu konular yer almaktadır\ldots{}

\hypertarget{bu-ikinci-seviye-bir-alt-baux15flux131k}{%
\subsection{Bu ikinci seviye bir alt başlık}\label{bu-ikinci-seviye-bir-alt-baux15flux131k}}

\hypertarget{Bolum5}{%
\chapter{Bölüm 4 Başlık}\label{Bolum5}}

\hypertarget{bu-bir-alt-baux15flux131k-1}{%
\section{Bu bir alt başlık}\label{bu-bir-alt-baux15flux131k-1}}

Bu bölümde şu konular yer almaktadır\ldots{}

\hypertarget{bu-ikinci-seviye-bir-alt-baux15flux131k-1}{%
\subsection{Bu ikinci seviye bir alt başlık}\label{bu-ikinci-seviye-bir-alt-baux15flux131k-1}}

\hypertarget{sonuuxe7}{%
\chapter*{Sonuç}\label{sonuuxe7}}
\addcontentsline{toc}{chapter}{Sonuç}

If we don't want Conclusion to have a chapter number next to it, we can add the \texttt{\{-\}} attribute.

\textbf{More info}

And here's some other random info: the first paragraph after a chapter title or section head \emph{shouldn't be} indented, because indents are to tell the reader that you're starting a new paragraph. Since that's obvious after a chapter or section title, proper typesetting doesn't add an indent there.

\hypertarget{kaynaklar}{%
\chapter*{Kaynaklar}\label{kaynaklar}}
\addcontentsline{toc}{chapter}{Kaynaklar}

Placeholder

\appendix

\hypertarget{ilk-ek-baux15flux131ux11fux131}{%
\chapter{İlk Ek Başlığı}\label{ilk-ek-baux15flux131ux11fux131}}

This first appendix includes all of the R chunks of code that were hidden throughout the document (using the \texttt{include\ =\ FALSE} chunk tag) to help with readibility and/or setup.

\textbf{In the main Rmd file}

\textbf{In Chapter \ref{ref-labels}:}

\hypertarget{ikinci-ek-baux15flux131ux11fux131}{%
\chapter{İkinci Ek Başlığı}\label{ikinci-ek-baux15flux131ux11fux131}}

İkinci Ek




\end{document}
